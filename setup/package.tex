%---------------------------------------------------------------------------------
%                学位论文LaTeX模板使用宏包文件
%---------------------------------------------------------------------------------

% 数学公式常用宏包(实现更多样的数学公式输入)
\usepackage{amsmath}
\newtheorem{myDef}{\hspace{2em}定义}[chapter]
\usepackage{mathtools}
\usepackage{wasysym}

\makeatletter  
\newif\if@restonecol  
\makeatother  
\let\algorithm\relax  
\let\endalgorithm\relax  
\usepackage[linesnumbered,ruled,vlined]{algorithm2e}%[ruled,vlined]{  
\usepackage{algpseudocode}  
\usepackage{amsmath}  
\renewcommand{\algorithmicrequire}{\textbf{Input:}}  % Use Input in the format of Algorithm  
\renewcommand{\algorithmicensure}{\textbf{Output:}} % Use Output in the format of Algorithm   
%\newcommand{\algorithmicbreak}{\textbf{break}}
\newcommand{\Break}{\textbf{break}}
\newcommand{\Retuan}{\textbf{return}}
\usepackage{algorithmicx}

\usepackage{colortbl}
%\usepackage{algorithm}  
%\usepackage{algpseudocode}  
%\usepackage{amsmath}  
%\renewcommand{\algorithmicrequire}{\textbf{Input:}}  % Use Input in the format of Algorithm  
%\renewcommand{\algorithmicensure}{\textbf{Output:}} % Use Output in the format of Algorithm 

\usepackage{listings}
\usepackage{xcolor}
\lstset{
	basicstyle=\ttfamily,
	numbers=left, 
	%numberstyle=\tiny, 
	keywordstyle=\color{blue!70},  
	commentstyle=\color{red!50!green!50!blue!50}, 
	frame=shadowbox, 
	rulesepcolor=\color{red!20!green!20!blue!20},
	tabsize=4,breaklines 
}
%\lstset{
%	basicstyle=\ttfamily,
%	numbers=left,
%	keywordsstyle=\color[RGB]{0,0,255},
%	commentstyle=\color[RGB]{0,128,0}
%	frame=shadowbox,
%	rulesepcolor=\color{red!20!green!20!blue!20},
%	showspaces=false,
%	showstringspaces=false,
%	showtabs=false,
%	tabsize=4,breaklines
%}


\usepackage{booktabs}
\usepackage{threeparttable}
\usepackage{amssymb}

\usepackage{amsfonts}
% 表格制作常用宏包(实现更多表格功能,比如不等线粗的三线表)
\usepackage{multirow}
\usepackage{booktabs}

% 插入图片常用宏包(实现多种格式的调用)
\usepackage{graphicx}

% 图表题注格式宏包(实现《规范》要求的题注格式)
\usepackage{caption}
\captionsetup{labelformat=simple, labelsep=space, font=bf}

% TeX系列标识的正确输入宏包(实现正确插入TeX相关的字符串)
\usepackage{dtklogos}

% 列表制作常用宏包(用于调用小间距列表)
\usepackage{paralist}
\usepackage{tabularx}
\usepackage{array}
% 外框宏包
\usepackage{framed}

% 引用文献实[Ni-Ni+j]的宏包
\usepackage{cite}

% 使用列表时要用到的包
\usepackage{enumitem}
% 英文采用Times New Rome字体(建议把这个宏包放在最后避免发生宏包冲突)
\usepackage{fontspec}
\setmainfont{Times New Roman}